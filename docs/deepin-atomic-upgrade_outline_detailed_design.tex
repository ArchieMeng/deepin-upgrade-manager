\documentclass{utart}

\usepackage{enumitem}
\usepackage{plantuml}
\usepackage{diagbox}
\usepackage{float}
\usepackage{listings}

\lstset{
    language=Json
    numbers=none,
    frame=single
}

\title{原子更新概要设计文档}
\author{liuchenghao}

\setUTClassify{C级商密}

\setUTVersion{V1.6}

% 设置文档编号
\setUTIndex{UT-YZGX20220224T_SYS026}

% 设置拟制人信息
\setUTFiction{刘成昊}{2022-03-23}

% 设置审核人信息
\setUTReview{闫博文}{2022-04-06}

% 设置批准人信息
\setUTApprove{闫博文}{2022-04-06}

\begin{document}
\utMakeTitle{}{1.0.0}{2022-02-22}
\utMakeChangeLog{
  1.0 & 创建 & 刘成昊 & 2022-03-23 \\
  \hline
  1.1 & 修改技术评审提出问题,详情见概要审计评审报告 & 刘成昊 & 2022-04-06 \\
  \hline
  1.2 & 关键数据结构中增加DBus接口描述 & 刘成昊 & 2022-06-15 \\
  \hline
  1.3 & 增加套件工具模块设计 & 刘成昊 & 2022-06-24 \\
  \hline
  1.4 & 修改技术评审提出问题,详情见概要审计评审报告 & 刘成昊 & 2022-06-30 \\
  \hline
  1.5 & 关键数据结构增加 grub 界面显示字段,补充 initrd 调用二进制流程 & 刘成昊 & 2022-07-22 \\
  \hline
  1.6 & 增加设置回滚与提交的配置文件接口,详情请看变更记录V1.6 & 刘成昊 & 2022-09-29 \\
  \hline
}
\utMakeTOC

\section{概述}
\subsection{目的}
本文档是针对 deepin 原子更新程序给出的系统概要设计文档,在本文档中,将给出 deepin 原子更新程序的设计原则、静态结构设计、动态流程设计、非功能性设计等内容。
\par
deepin 系统更新程序的设计与实现是基于系统更新的需求分析,总体上将结合结构化设计的方法与文字描述,给出程序结构化的详细设计,与需求分析内容相对应,以保证系统设计的严谨性与可实现性。
在结构化部分,本文档将主要采取 UML 语言的包图、类图、序列图等进行程序设计。
\par
本文档的适用读者为 deepin 原子更新程序的产品经理、设计人员、开发人员、测试人员以及后续维护人员。

\subsection{术语说明}
\begin{itemize}[leftmargin=4em]
\item ostree: 是一个用于对Linux操作系统进行版本更新的系统,它可以被视为 "面向操作系统二进制文件的git" 。通常用来做操作系统项目的持续交付;
\item hardlink: 硬链接,硬链接是通过指向同一个索引节点(inode)来进行文件链接,不支持跨分区或目录链接;
\item snapshot: 快照,是系统在某一时刻状态的副本;
\end{itemize}

\subsection{参考资料}
\begin{itemize}[leftmargin=4em]
\item \href{https://refspecs.linuxfoundation.org/FHS\_3.0/fhs/index.html}{Filesystem Hierarchy Standard.}
\item \href{https://ostree.readthedocs.io/en/latest/}{ostree}
\item \href{https://github.com/linuxdeepin/deepin-styleguide}{deepin 编码规范}
\end{itemize}

\section{系统设计}
deepin 系统更新程序最初的目标是将系统状态的变更操作,如软件的安装、更新或卸载,放入到 chroot 中操作;并提供系统快照的管理功能,如快照生成和快照回滚。
即通过快照构建隔离于当前系统运行环境的 chroot 环境,如此对系统的变更操作,便不会对当前的系统产生影响,仅在变更操作成功后才改变当前系统的状态。

当前阶段,基于时间和 deb 兼容的考虑,系统的变更仍在当前运行环境中进行,不构建 chroot 环境,但在系统变更前生成当前系统的快照。

\subsection{设计原则}
deepin 更新管理程序基于其需求,实现时需遵循以下原则:
\begin{itemize}[leftmargin=4em]
\item 更新程序应采用管道/过滤器与面向对象相结合的设计风格,流程控制主要采用管道/过滤器风格,功能模块主要采用面向对象风格;
\item 功能模块之间应避免相互调用,降低模块耦合度;
\item 模块应遵循最小化职责原则;
\item 实现时应尽可能只用系统库,禁止引入第三方库,因为需要集成进 \texttt{initrd.img} 中;
\item 执行时应实现失败处理逻辑,尽可能降低对现有系统的影响;
\item 应检查每个流程的产物,确认执行成功;
\item 应遵守 deepin 编码规范;
\end{itemize}

\subsection{子系统设计}
\subsubsection{结构设计}
deepin 更新程序被设计为一个单文件程序,通过命令行参数来执行不同的流程。
其中 init、commit、rollback、clean 等操作,应进行单例检查,不允许并行执行。

整体结构如下:
\begin{center}
  \begin{adjustbox}{scale=0.75}
    \begin{plantuml}
      @startuml
      !include /home/mecry/Documents/design/common/C4_Component.puml
      !include /home/mecry/Documents/design/common/C4_Container.puml

      Container(atomic, "deepin-atomic-upgrader", "Deepin Automic Upgrader") {
        Component(options, "Command Options")
        Component(ifc, "DBus Interface")
        Lay_R(options, ifc)
      }
      Container(kit, "deepin-boot-kit") {
        Component(kitoptions, "Kit Command Options")
        ComponentDb(spec, "Spec Config Pools")
        Lay_R(kitoptions, spec)
      }

      Boundary(pkgs, "Packages") {
        Boundary(controllers, "Controllers") {
          Component(init, "Init")
          Component(version, "Version")
          Component(commit, "Commit")
          Component(rollback, "Rollback")

          Lay_R(init, version)
          Lay_R(commit, rollback)
        }

        Boundary(modules, "Modules") {
          Component(repo, "Repo Manager") {
            Component(ostree, "OSTree Backend")
            Component(btrfs, "Btrfs Backend")
            Component(zfs, "ZFS Backend")

            Lay_R(ostree, btrfs)
            Lay_R(btrfs, zfs)
          }
          Component(fstab, "FSTab")
          Component(mount, "MountInfo")
          Component(rootfs, "Rootfs")
          Component(log, "Log")
          Component(config, "Configurations")

          Component(derivation, "Version Derivation")
          Container(bootMenuSpec, "Boot Menu Spec Parser")
          Container(rollbackSpec, "Rollback Spec Parser")
          Container(initrd, "initrd Spec Parser")

          Lay_R(config, log)
          Lay_R(log, derivation)
          Lay_D(config, fstab)
          Lay_R(fstab, mount)
          Lay_R(mount, bootMenuSpec)
          Lay_D(fstab, rootfs)
          Lay_R(rootfs, rollbackSpec)

          Lay_D(initrd, repo)
          Lay_D(derivation, bootMenuSpec)
          Lay_D(bootMenuSpec, rollbackSpec)
          Lay_D(rollbackSpec, initrd)
        }

        Lay_D(controllers, modules)
      }

      Rel_D(atomic, controllers, "uses")
      Rel_D(atomic, kit, "uses")
      Rel_D(kit, derivation, "uses")
      Rel_D(kit, bootMenuSpec, "uses")
      Rel_D(kit, rollbackSpec, "uses")
      Rel_D(kit, initrd, "uses")
      Rel_D(controllers, modules, "uses")
      @enduml
    \end{plantuml}
  \end{adjustbox}

  图 2-1 整体结构
\end{center}

更新程序主要由外部调用,内部模块,与引导管理三个部分组成。

\paragraph{外部调用}
deepin-atomic-upgrader 对外接口,由 dbus 接口与终端命令行组成。
\begin{itemize}[leftmargin=4em]
  \item Command Options: 更新程序支持用户在终端输入的调用;
  \item DBus Interface: 更新程序在被外部程序调用时通过dbus接口来实现;
\end{itemize}

\paragraph{引导管理}

更新程序的 grub 引导显示与 inird 回滚的被调用逻辑是通过 deepin-boot-kit 程序进行管理。

\paragraph{内部模块}
更新程序的内部实现是通过 Packages 来管理,其主要分外部调用接口与内部模块实现,其主要部分说明如下:

\begin{itemize}[leftmargin=4em]
  \item Commit: 提交当前系统至仓库的接口;
  \item Rollback: 回滚系统至指定版本接口;
  \item MountInfo: 获取当前系统实时分区挂载信;
  \item Boot Menu Spec Parser: 当前系统引导菜单的显示,有更新程序显示内容,deepin-boot-kit 来规定内容格式;
  \item Repo Manager: 更新程序内部的仓库管理,提供处理仓库的基础命令;
\end{itemize}

\subsubsection{引导套件设计}
deepin 引导套件被设计为一个管理grub的框架,使用其框架的所有工具需要遵循其规则。

整体结构如下:
\begin{center}
  \begin{adjustbox}{scale=0.75}
    \begin{plantuml}
    @startuml
    package "deepin-boot-kit" {
        usecase "spec pool"
        usecase "shell pool"

        [version generator] -[hidden]r-> [submenu spec]
        [spec pool] -r-> [submenu spec]

        [spec pool]-[hidden]l->[submenu generator]

        [submenu generator]

        [version generator] <-d- [submenu generator]
        [submenu spec] <-d- [submenu generator]

        [submenu generator] -[hidden]d-> [backup spec]

        [shell pool] -l-> [backup spec]
        [shell pool]-[hidden]l->[submenu generator]

    }

    package "initrd" {
        [initrd generator] --> [backup spec]

    }
  @enduml
\end{plantuml}
\end{adjustbox}

  图 2-2 套件程序结构
\end{center}

deepin-boot-kit 主要由 submenu 与 initrd 组成。

\paragraph{submenu}
submenu 为系统的引导子菜单,由 submenu generator 遵循 submenu spec 进行生成。
\paragraph{initrd}
initrd 为系统在init阶段实际运行的系统,其生成会读取 backup spec。

\subsubsection{版本名称设计}
deepin 更新程序的自身版本号由引导套件工具进行生成,更新程序按照固定接口进行增量获取,设计如下:
\begin{center}
  \begin{adjustbox}{scale=0.75}
    \begin{plantuml}
      @startuml
      package "version" {
      [version generator] --> [version spec]
     }
    sort -- [version generator]
    new -- [version generator]
    @enduml
  \end{plantuml}
\end{adjustbox}

图 2-3 版本名称
\end{center}

\begin{itemize}[leftmargin=4em]
  \item sort: 版本排序,按提交位次进行顺序排列;
  \item new: 版本生成,按照指定规则生成版本号;
  \item version spec: 版本生成规则;
  \item version generator: 版本生成;
\end{itemize}


\subsubsection{目录组织设计}

deepin 更新程序对系统文件的组成进行了重新设计,系统文件由快照, \texttt{/vendor} ,\texttt{/persistent}组合组成,结构如下:
\begin{center}
  \begin{adjustbox}{scale=0.75}
    \begin{plantuml}
    @startuml
    package Snapshot {
      [version1] -[hidden]r-> [version2]
      [version2] -[hidden]r-> [versionX]
    }

    [Repo] -u-> [version1]
    [Repo] -u-> [version2]
    [Repo] -u-> [versionX]

    [ActiveVersion] ..> [versionX]
    [ActiveVersion] -[hidden]r-> [/vendor]
    [OS] <-d- [/persistent]
    [OS] <-d- [/vendor]
    [OS] <-d- [ActiveVersion]

    note right of [Repo]
    位于 /persistent/osroot/repo
    end note
    note bottom of Snapshot
    位于 /persistent/osroot/snapshot
    end note
    @enduml
    \end{plantuml}
  \end{adjustbox}

  图 2-4 OS 组成
\end{center}

\begin{itemize}[leftmargin=4em]
  \item OS: 当前操作系统;
  \item ActiveVersion: 当前系统正在使用的仓库版本;
  \item /vendor: 第三方厂商软件安装的目录;
  \item /persistent: 系统数据分区;
  \item Repo: 更新程序的仓库,存放所有版本提交的文件;
  \item Snapshot: 仓库检出的指定版本快照,为仓库的硬链接;
\end{itemize}

\subsubsection{repo 设计}
repo 是 deepin 更新程序中最核心的对象,repo 采用了面向接口的设计方法,定义了 repo 必需的接口,并由具体的 repo 管理技术实现。

repo 理论上需要支持 ostree、btrfs、zfs 三种技术,现阶段只支持ostree。

repo 的组成结构如下:
\begin{center}
  \begin{adjustbox}{scale=0.55}
    \begin{plantuml}
      @startuml
      package Repo {
        [Interface]

        [ostree] -[hidden]r-> [btrfs]
        [btrfs] -[hidden]r-> [zfs]

        [Interface] <-d- [ostree]
        [Interface] <-d- [btrfs]
        [Interface] <-d- [zfs]

        NewRepo - [Interface]
      }
      @enduml
    \end{plantuml}
  \end{adjustbox}

  图 2-5 repo 结构
\end{center}

\subsection{关键流程设计}

\subsubsection{V20 升级至 V23}
\begin{center}
  \begin{adjustbox}{scale=0.65}
    \begin{plantuml}
      @startuml
      actor User
      User -> 升级工具 : 执行升级
      升级工具 -> AtomicManager : 通过接口设置备份  V23 系统配置文件
      升级工具 -> AtomicManager : 开始备份 V23 系统
      alt 备份失败
      AtomicManager --> 升级工具 : 最大分区不足存放仓库
      end
      alt 备份失败
      升级工具 --> User : 升级失败
      end
      升级工具 -> AtomicManager : 通过接口设置备份  V20 系统配置文件
      升级工具 -> AtomicManager : 开始备份 V20 系统
      alt 备份失败
      AtomicManager --> 升级工具 : 最大分区不足存放仓库
      end
      alt 备份失败
      升级工具 --> User : 升级失败
      end
      升级工具 -> AtomicManager : 回滚至刚备份的 V23 系统
      alt 回滚失败
      AtomicManager --> 升级工具 : 还原至原先系统
      end
      alt 回滚失败
      升级工具 --> User : 升级失败
      end
      AtomicManager --> 升级工具 : 回滚成功
      升级工具 -> AtomicManager : 通过接口还原默认备份配置文件
      升级工具 --> User : 升级成功
      @enduml
    \end{plantuml}
    \end{adjustbox}

    图 2-6 V20升级V23时序图
\end{center}
\begin{itemize}[leftmargin=4em]
  \item AtomicManager 为原子更新程序, 此图指 V20 升级 V23 的更新程序;
  \item 升级工具利用原子更新的 setdefaultconfig 接口来调整配置文件信息,其具体细节查看章节 2.3.3.2;
  \item 升级工具在升级过程中若失败则还原系统;
\end{itemize}

\subsubsection{V23 安装时初始化仓库}
\begin{center}
  \begin{adjustbox}{scale=0.65}
    \begin{plantuml}
      @startuml
      actor User
      User -> DeepinInstaller : 执行安装
      DeepinInstaller -> DeepinInstaller : 释放文件,oem文件安装
      DeepinInstaller -> AtomicUpdate : 调用设置备份路径配置接口(全盘路径)
      DeepinInstaller -> AtomicUpdate : 调用接口初始化仓库
      AtomicUpdate --> DeepinInstaller : 返回仓库初始化结果
      alt 初始化失败
      DeepinInstaller --> User : 仓库初始化失败
      end
      DeepinInstaller -> AtomicUpdate : 调用设置备份路径配置接口(默认配置)
      DeepinInstaller -> DeepinInstaller : 设置用户账户密码
      DeepinInstaller --> User : 进入系统
      @enduml
      \end{plantuml}
    \end{adjustbox}

    图 2-7 V23 安装器时序图
\end{center}
\begin{itemize}[leftmargin=4em]
  \item AtomicManager 为原子更新程序,DeepinInstaller 为系统安装器;
  \item 在安装器将 oem 安装包安装完成后与设置用户账户密码前调用原子更新初始化仓库接口;
  \item 在安装器进行安装时,新增是否需要做初始化备份的选项;
  \item 当更新程序初始化仓库失败,不会影响到安装器主流程;
  \item 默认配置指为原子更新自带 V23 配置文件,通过 \texttt{deepin-upgrade-manager --action=setdefaultconfig} 进行设置,其具体细节查看章节 2.3.3.2;
\end{itemize}

\subsubsection{设置提交回滚配置文件}
\paragraph{触发场景}
\begin{itemize}[leftmargin=4em]
  \item V20升级V23时会首先设置配置文件,原子更新根据配置文件进行提交;
  \item 安装器初始化备份时,设置配置文件,原子更新根据配置文件进行提交;
\end{itemize}
\paragraph{逻辑流程}
setdefaultconfig 不同场景触发流程一致,其流程如下:
\begin{center}
  \begin{adjustbox}{scale=0.65}
    \begin{plantuml}
      @startuml
      start
      :传入配置路径设置原子更新配置文件;
      if(路径是否存在)then(否)
      :退出;
      stop
      else(是)
      if(配置文件是否与当前一致)then(是)
      :退出;
      stop
      else(否)
      :对配置文件保存至指定目录;
      stop
      @enduml
    \end{plantuml}
  \end{adjustbox}

    图 2-8 设置默认配置
\end{center}
\begin{itemize}[leftmargin=4em]
  \item 当仓库不存在时,调用此接口将配置存放至本地目录 /var/lib/deepin-upgrade-manager/config/ready/,若仓库已存在则直接将其存放至仓库目录。
  \item 此功能可以通过dbus接口或命令行 \texttt{deepin-upgrade-manager --action=setdefaultconfig --data=./data.yaml} 进行设置,命令行中不增加 \texttt{--data}参数时默认路径为 V23 的备份数据。
\end{itemize}

\subsubsection{repo 初始化}
\paragraph{触发场景}
\begin{itemize}[leftmargin=4em]
  \item V20升级V23时会首先初始化此仓库,对文件进行备份;
  \item 安装器安装系统时会提供是否初始化仓库选项,若点击则对仓库进行初始化;
  \item 若仓库未初始化则控制中心提供初始化仓库按钮,用户可以从控制中心进行初始化仓库;
\end{itemize}
\paragraph{逻辑流程}
init 不同场景触发流程一致,其流程如下:
\begin{center}
  \begin{adjustbox}{scale=0.65}
    \begin{plantuml}
      @startuml
      start
      :开始初始化;
      if(repo 是否存在)then(是)
      :退出;
      stop
      else(否)
      :获取当前系统最大分区;
      :计算仓库需要空间大小;
      if(最大分区是否能存放仓库)then(否)
      :退出;
      stop
      else(是)
      :将配置文件拷贝至仓库目录;
      note right:/persistent/osroot/config/
      :修改配置文件中仓库路径;
      :在仓库目录新建本次提交的配置信息;
      note right:/persistent/osroot/config/v23.0.0.2022/
      :version commit;
      if(仓库路径是否在备份路径中)then(否)
      else(是)
      :仓库路径写入至配置文件过滤路径中;
      endif
      :更新grub模块;
      :系统重启显示启动项;
      stop
      @enduml
    \end{plantuml}
  \end{adjustbox}

  图 2-9 repo init
\end{center}
\begin{itemize}[leftmargin=4em]
  \item 配置文件目录的优先级为 /var/lib/deeepin-upgrade-manager/config/ 大于 /etc/deeepin-upgrade-manager/;
\end{itemize}

\subsubsection{rootfs 数据准备}
rootfs 文件夹为系统提交所需内容,其挂载至与 /usr 同级分区中;
\paragraph{触发场景}
\begin{itemize}[leftmargin=4em]
  \item 当 version 提交时会触发此动作;
\end{itemize}
\paragraph{逻辑流程}
\begin{center}
  \begin{adjustbox}{scale=0.65}
    \begin{plantuml}
      @startuml
      start
      :提交时触发数据准备;
      :获取 /usr 文件夹挂载分区;
      if(判断分区空间大小,是否能够存放rootfs)then(是)
      else(否)
      :退出报错;
      stop
      endif
      if(rootfs文件夹是否存在)then(是);
      :删除rootfs文件夹;
      else(否);
      endif

      :在/usr 同分区下新建rootfs文件夹;
      :遍历待备份目录文件;
      :rootfs 文件与待备份文件进行对比;
      if(是否在同分区下?)then(是)
      :硬链接;
      else(否)
      :拷贝;
      if(空间是否不足)then(是)
      :删除rootfs文件夹;
      :退出并报错;
      stop;
      else(否)
      endif
      endif
      :rootfs文件准备完成;
      stop;
      @enduml
    \end{plantuml}
  \end{adjustbox}

  图 2-10 rootfs 文件准备完成
\end{center}

\begin{itemize}[leftmargin=4em]
  \item rootfs 由于其内容为系统文件硬链接,所以位置必须与 /usr 分区相同;
  \item ostree 支持软链提交,并在回滚时会保留软链;
\end{itemize}

\subsubsection{version 提交}
\paragraph{触发场景}
\begin{itemize}[leftmargin=4em]
  \item 当 repo 初始化时 sudo deepin-upgrade-manager --action=init,默认会触发一次全
  量提交;
  \item 控制中心提供备份系统接口,手动点击时会触发此动作;
\end{itemize}
\paragraph{逻辑流程}
commit 不同场景触发流程一致,其流程如下:
\begin{center}
  \begin{adjustbox}{scale=0.65}
    \begin{plantuml}
      @startuml
      start
      :开始提交;
      :通过套件工具获取版本号;
      :新建对应版本号的配置文件夹;
      note right:/persistent/osroot/config/V23.0.0.2022/
      :将 ready 目录下的配置文件拷贝至配置文件夹中;
      :通过 ready 目录下配置文件生成 config.json;
      :解析config.json获取提交目录列表;
      note right: 循环读取config.json,其中 subscribe-list 为需要提交目录
      :rootfs文件准备;
      :提交rootfs文件;
      :显示提交版本号;
      :更新grub;
      :提交完成;
      stop;
      @enduml
    \end{plantuml}
  \end{adjustbox}

  图 2-11 repo commit
\end{center}

\begin{itemize}[leftmargin=4em]
  \item 本次 config.json (\texttt{/persistent/osroot/config/config.json}) 是由 ready目录下的配置生成(\texttt{/persistent/osroot/config/ready/data.yaml})。
  \item 读取 config.json 配置文件,内容为待备份文件路径,仓库版本,快照路径,临时存储文件路径;
  \item 在提交前需要将之前准备的配置文件备份至与版本同名的配置文件夹下;
  \item 在提交结束后,将原子更新的 defaultconfig 路径设置为本次提交版本下的配置文件路径;
\end{itemize}
\paragraph{错误处理}
\begin{description}[leftmargin=!]
  \item[更新rootfs错误:] : 在每次进行更新时会删除 rootfs 文件夹,再进行重新生成;
\end{description}

\subsubsection{version 回滚}
\paragraph{触发场景}
\begin{itemize}[leftmargin=4em]
  \item 当用户在grub界面选择时会进行触发;
  \item 当备份还原选择回滚时,会进行版本回滚;
\end{itemize}
\paragraph{逻辑流程}
rollback 不同场景触发流程一致,其流程如下:
\begin{center}
  \begin{adjustbox}{scale=0.75}
    \begin{plantuml}
      @startuml
      start
      :开始回滚;
      if(当前是否在grub等待界面中)then(是)
      if(当前 grub 是否存在加密) then (是)
      :输入grub管理员账户密码;
      else
      endif
      :选择指定回滚版本;
      else(否)
      :设置回滚状态下次重启后回滚;
      :重启;
      if(当前 grub 是否存在加密) then (是)
      :输入grub管理员账户密码;
      else
      endif
      endif
      :挂载仓库分区;
      :获取当前挂载根系统的目录;
      if(确认本地挂载是否与仓库中/etc/fstab挂载点相同)then(不相同)
      :进行重新挂载;
      else(相同)
      endif
      :根据版本号匹配回滚所需配置文件;
      note left:/persistent/osroot/config/V23.0.0.2022/data.yaml
      :生成本次回滚配置;
      note left:/persistent/osroot/config/config.json
      :检出指定版本至<b>快照目录</b>;
      note left: 与 repo 同一分区
      :遍历待回滚目录;
      if(子目录是否与父目录挂载分区相同?)then(是)
      :回滚目录同级创建<b>恢复目录</b>;
      :遍历<b>快照目录</b>文件并与待回滚目录中文件进行比对;
      else(否)
      :子目录下创建<b>恢复目录</b>;
      :遍历<b>快照目录</b>文件并与子目录中文件进行比对;
      endif
      if(是否为相同文件)then(是)
      :硬链接至<b>恢复目录</b>;
      note left: 将待回滚目录文件硬链接至<b>恢复目录</b>
      else(否)
      :拷贝至<b>恢复目录</b>;
      note right: 将<b>快照目录</b>文件拷贝至<b>恢复目录</b>
      endif
      :<b>恢复目录</b>替换待恢复目录;
      note left: 跨分区时,需要将<b>恢复目录</b>文件全量移至父目录,并将父目录文件删除
      :回滚完成;
      stop;
      @enduml
    \end{plantuml}
  \end{adjustbox}

  图 2-12 repo version rollback
\end{center}

\begin{itemize}[leftmargin=4em]
  \item 恢复目录: 系统所需的待回滚版本文件,快照目录:  与 repo 同分区目录,内容为仓库检出指定回滚版本硬链接;
  \item 回滚首先会判断当前系统状态,是否在 grub 选择界面中,若在其中就会识别 grub 参数进行回滚动作,若不存在则设置回滚状态,重启后回滚;
  \item 在创建恢复目录时,为了节省空间会选择与待恢复目录同一分区进行创建。
  \item 回滚时会通过状态文件中分区 UUID,或 grub 参数中分区的 UUID ,来挂载仓库分区;
\end{itemize}

\paragraph{错误处理}
\begin{description}[leftmargin=!]
  \item[根路径未获取到:] 退出程序并进行报错处理;
  \item[在 initramfs 回滚时异常:] 在临时目录替换待恢复目录时,将 /boot 目录设置为最后替换,防止失败时无法找到引导项;
\end{description}

\subsubsection{version 清理}
\paragraph{触发场景}
\begin{itemize}[leftmargin=4em]
  \item 当控制中心选择清理时,会进行版本清理;
\end{itemize}

\paragraph{逻辑流程}
clean 不同场景触发流程一致,其流程如下:
\begin{center}
  \begin{adjustbox}{scale=0.65}
    \begin{plantuml}
      @startuml
      :传入 clean 参数调用更新程序;
      if (是否已有更新程序运行) then (Y)
      :显示错误信息;
      else
      :列出所有 version;
      if (version 个数是否大于最大保留数量) then (N)
      :显示提示信息;
      else
      :生成待清理版本列表;
      note right:初始版本不显示
      while (遍历待清理版本列表)
      :获取版本对应的提交列表;
      :删除版本;
      while (遍历提交列表)
      :删除提交;
      endwhile
      endwhile
      endif
      endif
      stop
      @enduml
    \end{plantuml}
  \end{adjustbox}

  图 2-13 repo version clean
\end{center}
\begin{itemize}[leftmargin=4em]
  \item 版本清理时需要将 ostree 的 branch 中 commit 进行全部删除;
  \item 最大保留数量在控制中心可以设置;
  \item 生成待清理版本时不显示初始版本,防止无法还原初始系统;
\end{itemize}

\subsubsection{版本号获取}
\paragraph{触发场景}
\begin{itemize}[leftmargin=4em]
  \item 当更新程序提交当前系统时调用套件工具进行版本获取;
\end{itemize}

\begin{center}
  \begin{adjustbox}{scale=0.75}
    \begin{plantuml}
      @startuml
      :传入 version 参数调用引导套件程序;
      if (存储版本信息文件不存在) then (Y)
      :创建存储版本文件;
      :创建初始化版本号;
      else(N)
      :读取数据文件;
      :进行增量计算;
      endif
      :版本信息写入本地存储文件;
      :返回新增版本号;
      stop
      @enduml
    \end{plantuml}
  \end{adjustbox}

  图 2-14 new version
\end{center}
\begin{itemize}[leftmargin=4em]
  \item 获取版本号通过命令:deepin-boot-kit --action=version 进行增量获取;
\end{itemize}

\subsubsection{引导更新}
\paragraph{触发场景}
\begin{itemize}[leftmargin=4em]
  \item 当系统进行版本提交时会触发,显示最新版本;
  \item 当系统进行版本清理会触发,已清理的版本不会进行显示;
  \item 当系统进行版本回滚时触发,显示回滚后的版本;
\end{itemize}

\begin{center}
  \begin{adjustbox}{scale=0.75}
    \begin{plantuml}
      @startuml
      :传入 update 参数调用引导套件程序;
      :更新本地grub;
      note left:update-grub
      :循环加载配置;
      note left:在 /var/lib/deepin-upgrade-manager/config/目录
      :通过配置获取grub生成必要信息;
      :获取当前系统引导加密账户;
      if (是否存在引导加密账户) then(是)
      :引导增加参数进行加密设置;
      else
      endif
      :grub文件写入;
      stop
      @enduml
    \end{plantuml}
  \end{adjustbox}

  图 2-15 update grub
\end{center}
\begin{itemize}[leftmargin=4em]
  \item 更新本地 grub 时,会触发脚本调用,使引导程序获取 grub 需要显示的信息;
  \item grub 加密必须在控制中心设置 grub 管理员账户密码并对此功能进行开启才可以使用加密功能;
  \item 通过 com.deepin.daemon.Grub2.EditAuthentication的dbus 接口获取当前系统引导加密账户;
  \item 配置详情见关键数据设计;
\end{itemize}

\subsubsection{initrd 更新}
\paragraph{触发场景}
\begin{itemize}[leftmargin=4em]
  \item 修改initramfs下的脚本会自动出发initrd更新;
\end{itemize}

\begin{center}
  \begin{adjustbox}{scale=0.75}
    \begin{plantuml}
      @startuml
      :本地 initramfs 被更新;
      :调用 initrd 生成脚本;
      :调用套件工具;
      :循环加载配置信息;
      note left:在 /var/lib/deepin-upgrade-manager/config/目录
      :通过配置获取initrd生成必要信息;
      :initrd生成;
      stop
      @enduml
    \end{plantuml}
  \end{adjustbox}

  图 2-16 update initrd
\end{center}
\begin{itemize}[leftmargin=4em]
  \item 在 update-initramfs 出发自身的 hook 脚本,从而调用套件工具;
  \item 必要信息包含工具中运行二进制的脚本路径,在initrd生成时将其进行拷贝;
  \item 配置详情见关键数据设计;
\end{itemize}

\subsubsection{initrd 调用二进制}
\begin{center}
  \begin{adjustbox}{scale=0.65}
    \begin{plantuml}
      @startuml
      :用户在grub选择指定版本;
      if(当前 grub 是否存在加密) then (是)
      :输入grub管理员账户密码;
      else
      endif
      :进入系统initrd;
      note left:init-bottom阶段
      :读取内核参数;
      :将回滚版本号设置为环境变量;
      note left:export ROLLBACK_VERSION=
      :获取指定工具脚本的路径;
      :遍历指定路径下的脚本进行排序;
      :运行排序后的脚本;
      stop
      @enduml
    \end{plantuml}
  \end{adjustbox}

  图 2-17 initrd runing tool
\end{center}


\begin{itemize}[leftmargin=4em]
  \item 在 initrd 中的脚本调用通过传入的内核参数的 back\_scheme 来确认需要运行的脚本文件夹,文件夹中包含工具回滚的所有操作,由工具进行定义;
  \item 排序为数字排序,数字越小优先级越高;
  \item 工具脚本文件夹中存放回滚的具体流程;
  \item 工具脚本通过内核参数cmdline中的 back\_version 或 deepin-boot-kit设置 ROLLBACK\_VERSION 环境变量来获取具体回滚版本;
\end{itemize}

\subsection{数据结构设计}
\subsubsection{版本号}
\paragraph{版本号格式}
本地仓库的版本号由 \texttt{<distribution>.<major>.<minor>.<date>} 组成。
\texttt{distribution} 是系统大版本,只允许包含字母和数字,字母全小写,与 \texttt{debian} 中的含义保持一致;
\texttt{major} 和 \texttt{minor} 是无符号整数,\texttt{major} 在 \texttt{date} 发生变化时自增,并将 \texttt{minor} 重置;
\texttt{minor} 则是在 \texttt{date} 不变时自增, \texttt{major} 则保持不变;
\texttt{date} 是当前的时间,由 \texttt{YearMonthDay}组成,如 \texttt{20220222} 。

这里用下面的例子说明版本号的变更方式:

\begin{quote}
假设 \texttt{distribution} 为 \texttt{V23} ,\texttt{date} 为 \texttt{20220222} 。

则初始化本地仓库时,版本号则为 \texttt{V23.0.0.20220222} ,在日期不变时,对系统做了一些修改,再次提交时,版本则变为 \texttt{V23.0.1.20220222} 。

而在一天后,再次提交时,版本号则为 \texttt{V23.1.0.20220223}。
\end{quote}
\paragraph{版本号比较逻辑}
\begin{quote}
Example 1 : 假设 distribution 为 V23,在 2022 年 4 月 1 日安装好系统进行了两
次提交,则两次的版本号分别为 V23.0.1.20220401 和 V23.0.2.20220401 此时则可
比较 minor 的大小。\\

Example 2 : 假设 distribution 为 V23,在 2022 年 4 月 1 日安装好系统进行了一
次提交,在 2022 年 4 月 5 日安装好系统进行了一次提交,则两次的版本号分别为
V23.0.0.20220401 和 V23.1.0.20220405 此刻可以比较 major 大小。
\end{quote}
\subsubsection{Repo}
Repo 中定义了仓库管理的接口,OSTree 进行实现,数据结构如下:
\begin{center}
  \begin{adjustbox}{scale=0.75}
    \begin{plantuml}
      @startuml
      package Repo {
        interface Repository {
          Init() error
          Exist(string version)(bool)
          Last() (string, error)
          List() ([]string, error)
          Snapshot(version, snapDir string) error
          Commit(version, subject, dataDir string) error
          Diff(baseVersion, targetVersion, dstFile string) error
          Cat(version, filepath, dstFile string) error
          Previous(version string) (string, error)
        }
        class OSTree implements Repository {
          - repoDir string
        }
      }
      @enduml
    \end{plantuml}
  \end{adjustbox}

  图 2-18 repo object
\end{center}
\begin{itemize}[leftmargin=4em]
  \item Init : 初始化仓库,并进行一次全量提交;
  \item Exist : 判断仓库是否存在此版本;
  \begin{itemize}[leftmargin=4em]
    \item version : 字符串类型,传入需判断版本号;
    \item return : bool 表示此版本是否存在,若存在则为 true ,不存在则为 false;
  \end{itemize}
  \item Last : 获取最后一次提交时版本;
  \begin{itemize}[leftmargin=4em]
    \item return : 字符串类型,返回最后一次提交版本;
  \end{itemize}
  \item List: 获取当前仓库所有版本号;
  \begin{itemize}[leftmargin=4em]
    \item return : 字符串数组类型,返回的当前仓库版本号列表;
  \end{itemize}
  \item Snapshot: 获取指定版本快照;
  \begin{itemize}[leftmargin=4em]
    \item version : 字符串类型,指定版本号;
    \item snapDir : 字符串类型,获取快照时,存放本地路径;
  \end{itemize}
  \item Commit: 提交文件至仓库;
  \begin{itemize}[leftmargin=4em]
    \item version : 字符串类型,指定版本号;
    \item subject : 字符串类型,此次提交的主题;
    \item dataDir : 此次提交文件路径;
  \end{itemize}
  \item Diff: 比较两次提交,并输出不同;
  \begin{itemize}[leftmargin=4em]
    \item baseVersion : 字符串类型,待比较版本号;
    \item targetVersion : 字符串类型,需比较版本号;
    \item dstFile : 字符串类型,两版本中不同内容;
  \end{itemize}
  \item Cat: 获取指定版本,指定内容;
  \begin{itemize}[leftmargin=4em]
    \item version : 字符串类型,指定版本号;
    \item filepath : 字符串类型,指定版本文件中文件路径;
    \item dstFile : 字符串类型,获取指定分支指定文件内容;
  \end{itemize}
  \item Previous: 获取指定版本上一个版本号;
  \begin{itemize}[leftmargin=4em]
    \item version : 字符串类型,指定版本号;
    \item return : 字符串类型,返回上一个版本号;
  \end{itemize}
\end{itemize}

\subsubsection{DBus}
DBus 中定义了外部调用的接口。
\begin{itemize}[leftmargin=4em]
  \item service name:org.deepin.AtomicUpgrade1
  \item object path:/org/deepin/AtomicUpgrade1
  \item interface name:org.deepin.AtomicUpgrade1
\end{itemize}

\paragraph{Methods}
\begin{itemize}[leftmargin=4em]
  \item Commit (IN String  subject): 提交当前系统至本地仓库,异步调用,立即返回,若已存在提交或回滚进程则返回dbus错误;
  \item Rollback (IN String  version): 回滚系统至指定版本,异步调用,立即返回,若已存在提交或回滚进程则返回dbus错误;
  \item Delete (IN String  version): 删除仓库指定版本,当前激活版本与初始版本无法删除;
  \item ListVersion (OUT Array<String> list): 获取当前仓库的版本列表;
  \item QuerySubject (IN Array<String> versions)(OUT Array<String> list): 获取指定版本列表对应主题信息;
  \item SetDefaultConfig (IN String  path): 设置原子更新提交与回滚的默认配置文件;
  \item GetGrubTitle (IN String  version, OUT String  grubtitle):获取指定版本的引导展示信息;
  \item CancelRollback(): 取消系统回滚状态,异步调用,立即返回,若已存在提交,删除,回滚或重置操作则返回dbus错误;
\end{itemize}

\paragraph{Properties}
\begin{itemize}[leftmargin=4em]
  \item ActiveVersion: 当前系统中正在使用仓库的系统版本;
  \item DefaultConfig: 当前系统中下次提交时配置文件路径;
  \item Running: 当前程序是否在运行;
  \item RepoUUID: 当前仓库存储分区的UUID;
\end{itemize}


\begin{itemize}[leftmargin=4em]
  \item StateChanged (Int32 operate, Int32 state, String version,String message): 当前程序改变状态信号通知;
\end{itemize}

\subsubsection{原子更新 data.yaml}
data.yaml 为原子更新生成回滚与提交的数据的配置文件(config.json),数据格式为yaml,默认使用v23的配置,其存放路径为\texttt{/persistent/osroot/config/config/v23.0.0.20220302/data.yaml},原子更新默认自带 V23 配置其路径为 \texttt{/etc/deepin-upgrade-manager/ready/data.yaml}。
\paragraph{示例}
\begin{lstlisting}
  target:
    backup_list:
        - "/boot"
        - "/usr"
        - "/var"
        - "/etc/pam.d"
        - "/etc/os-version"
        - "/etc/os-release"
        - "/etc/apt/sources.list"
    hold_list:
        - "/opt"
        - "/etc"
        - "/usr/lib/dpkg-db"
        - "/var/spool"
        - "/var/lib/lightdm"
\end{lstlisting}

\paragraph{说明}
\begin{itemize}[leftmargin=4em]
  \item backup\_list: 需要备份文件路径;
  \item hold\_list: 需要备份与提交时需要保留原系统的文件路径;
\end{itemize}

\subsubsection{原子更新 config.json}
config.json 为原子更新的配置文件,其结构中包含生成,提交,回滚操作所需要的基本数据,数据格式为json,存放路径为\texttt{/etc/deepin-upgrade-manager/config.json},在仓库初始化后会使用仓库下的config.json,其路径为\texttt{/persistent/osroot/config/config.json}。

\paragraph{示例}
\begin{lstlisting}
  {
    "config_version": "1.0.10",
    "distribution": "v23",
    "active_version": "20220210",
    "cache_dir": "/usr/.osrepo-cache",
    "auto_cleanup": true,
    "max_repo_retention": 3,
    "max_version_retention": 5,
    "repo_list": [
      {
        "repo_mount_point":"/persistent",
        "repo": "/persistent/osroot/repo",
        "config_dir": "/persistent/osroot/config",
        "stage_dir": "/persistent/osroot/cache",
        "snapshot_dir": "/persistent/osroot/snapshot",
        "data_origin":"/etc/deepin-upgrade-manager/ready/data.yaml",
        "subscribe_list": [
          "/boot",
          "/usr",
          "/etc",
          "/var/lib/apt",
          "/var/lib/dkms",
          "/var/lib/dpkg",
          "/var/lib/man-db",
          "/var/lib/initramfs-tools",
          "/var/lib/systemd/deb-systemd-helper-enabled",
          "/var/lib/selinux"
        ],
        "filter_list": [
          "/usr/lib/locale/locale-archive",
          "/etc/locale.gen",
          "/usr/share/deepin-defender/localcache.db"
        ]
      }
    ]
  }
\end{lstlisting}

\paragraph{说明}
\begin{itemize}[leftmargin=4em]
  \item config\_version: 当前原子更新版本;
  \item distribution: 配置文件使用版本;
  \item cache\_dir: 当前释放系统文件缓存地址,一般会挂载至根分区;
  \item active\_version: 当前系统使用仓库文件的激活版本;
  \item max\_repo\_retention: 最大保存版本数量;
  \item repo\_list: 数组存储 repo 内部信息;
    \begin{itemize}[leftmargin=4em]
    \item repo\_mount\_point: 当前仓库存在分区;
    \item repo: repo仓库路径;
    \item config\_dir: 每个版本对应的配置文件所存放路径;
    \item snapshot\_dir: 快照存放路径;
    \item data\_origin: 备份路径与过滤路径的信息来源文件路径;
    \item subscribe\_list: 需要备份文件路径;
    \item filter\_list: 需要备份与提交时需要过滤的文件路径;
    \end{itemize}
\end{itemize}

\subsubsection{引导套件 config.json}
config.json 结构中包含引导工具需要生成initrd与grub的信息,数据格式为 json,文件由提交工具进行提供,格式由引导套件进行规定,其存放路径为/var/lib/deepin-boot-kit/config/,其文件命名应遵循<scheme>.json规则。

\paragraph{示例}
\begin{lstlisting}
  {
    "submenu":{
        "version_list": "deepin-upgrade-manager --action=bootlist"
      }
    "initrd":{
        "script_path": "/var/lib/deepin-upgrade-manager/script/"
      }
  }
\end{lstlisting}

\paragraph{说明}
\begin{itemize}[leftmargin=4em]
  \item submenu: grub 生成信息;
    \begin{itemize}[leftmargin=4em]
    \item version\_list: 获取 Version List 的具体命令;
    \end{itemize}
  \item initrd: initrd 生成信息;
    \begin{itemize}[leftmargin=4em]
    \item script\_path: 回滚工具存放脚本路径;
    \end{itemize}
\end{itemize}

\subsubsection{引导套件 Version List}
Version List 为指定工具提供的版本信息列表,由还原工具指定获取列表方法,由套件工具决定获取内容格式及必要字段;

\paragraph{示例}
\begin{lstlisting}
  {
    "version_list":[
      {
        "version": "v20.0.0.20220323",
        "kernel": "/boot/snapshot/v20.0.0.20220323/vmlinuz",
        "initrd": "/boot/snapshot/v20.0.0.20220323/initrd.img",
        "scheme": "atomic",
        "display": "deepin 23.1(2022/06/27 10:00:00)"
      }
    ]
  }
\end{lstlisting}

\paragraph{说明}
\begin{itemize}[leftmargin=4em]
  \item version\_list: 数组存储 version 内部信息;
    \begin{itemize}[leftmargin=4em]
    \item version: 版本名称;
    \item kernel: 当前版本内核存放路径,必须为系统启动挂载/boot后的文件路径;
    \item initrd: 当前版本initrd存放路径,必须为系统启动挂载/boot后的文件路径;
    \item scheme: 当前工具的类型,用于区分不同工具;
    \item display: grub 显示信息;
    \end{itemize}
\end{itemize}

\section{非功能性设计}
\subsection{性能}
\subsubsection{性能测试}
本次测试内容为初始化 repo ,二次提交三次提交仓库,在 HDD 和 SSD 下,在跨分区(跨分区: 将/boot,/var,/usr/local挂载至其他分区)和非跨分区的不同情况下表现,其测试结果如下:
\begin{table}[H]
  \begin{tabular}{|c|ccc|}
  \hline
             & \multicolumn{3}{c|}{i9-10885H/8GB/4核/SSD/虚拟机(全盘)}                          \\ \hline
             & \multicolumn{1}{c|}{首次(包含初始化repo)} & \multicolumn{1}{c|}{二次(重启后)} & 三次(重启) \\ \hline
  备份时间       & \multicolumn{1}{c|}{96}            & \multicolumn{1}{c|}{86}      & 88     \\ \hline
  准备rootfs时间 & \multicolumn{1}{c|}{11}            & \multicolumn{1}{c|}{9}       & 11     \\ \hline
  otree提交时间  & \multicolumn{1}{c|}{82}            & \multicolumn{1}{c|}{76}      & 74     \\ \hline
  删除缓存时间     & \multicolumn{1}{c|}{3}             & \multicolumn{1}{c|}{3}       & 3      \\ \hline
  \end{tabular}
  \end{table}
\begin{table}[H]
  \begin{tabular}{|c|ccc|}
  \hline
              & \multicolumn{3}{c|}{i7-10700/8GB/4核/HHD/虚拟机(全盘)}                           \\ \hline
              & \multicolumn{1}{c|}{首次(包含初始化repo)} & \multicolumn{1}{c|}{二次(重启后)} & 三次(重启) \\ \hline
  备份时间       & \multicolumn{1}{c|}{379}           & \multicolumn{1}{c|}{296}     & 299    \\ \hline
  准备rootfs时间 & \multicolumn{1}{c|}{75}            & \multicolumn{1}{c|}{75}      & 82     \\ \hline
  otree提交时间  & \multicolumn{1}{c|}{294}           & \multicolumn{1}{c|}{211}     & 210    \\ \hline
  删除缓存时间     & \multicolumn{1}{c|}{9}             & \multicolumn{1}{c|}{7}       & 7      \\ \hline
  \end{tabular}
  \end{table}
\begin{table}[H]
  \begin{tabular}{|c|ccc|}
  \hline
              & \multicolumn{3}{c|}{i9-10885H/8GB/4核/SSD/虚拟机(跨分区)}                         \\ \hline
              & \multicolumn{1}{c|}{首次(包含初始化repo)} & \multicolumn{1}{c|}{二次(重启后)} & 三次(重启) \\ \hline
  备份时间       & \multicolumn{1}{c|}{115}           & \multicolumn{1}{c|}{109}     & 107    \\ \hline
  准备rootfs时间 & \multicolumn{1}{c|}{34}            & \multicolumn{1}{c|}{38}      & 38     \\ \hline
  otree提交时间  & \multicolumn{1}{c|}{78}            & \multicolumn{1}{c|}{67}      & 65     \\ \hline
  删除缓存时间     & \multicolumn{1}{c|}{3}             & \multicolumn{1}{c|}{4}       & 3      \\ \hline
  \end{tabular}
  \end{table}
本次测试内容为在 HDD 和 SSD 不同情况下解压 systemfiles.squashfs 下表现,其测试结果如下:
\begin{table}[H]
  \begin{tabular}{|c|cc|cc|}
  \hline
   & \multicolumn{2}{c|}{i9-10885H/8GB/4核/SSD/虚拟机} & \multicolumn{2}{c|}{i7-10700/8GB/4核/HHD/虚拟机} \\ \hline
                          & \multicolumn{1}{c|}{第一次} & 第二次 & \multicolumn{1}{c|}{第一次} & 第二次 \\ \hline
  repo预制作 & \multicolumn{1}{c|}{42}  & 43  & \multicolumn{1}{c|}{280} & 310 \\ \hline
  无repo   & \multicolumn{1}{c|}{43}  & 41  & \multicolumn{1}{c|}{140} & 121 \\ \hline
  \end{tabular}
  \end{table}
\begin{table}[H]
  \begin{tabular}{|c|cc|}
  \hline
                          & \multicolumn{2}{c|}{i7-10700/8GB/4核/HHD/物理机} \\ \hline
                          & \multicolumn{1}{c|}{第一次}        & 第二次        \\ \hline
  repo预制作 & \multicolumn{1}{c|}{118}        & 122        \\ \hline
  无repo   & \multicolumn{1}{c|}{65}         & 86         \\ \hline
  \end{tabular}
  \end{table}

\subsubsection{性能分析}
\begin{itemize}[leftmargin=4em]
\item 4 核/SSD(增加 1 分半左右,时间增长 70\% 左右),4 核心/HDD(增加 5 分钟+,时间增长 55\% 左右);
\item 备份时间主要受 ostree 提交操作影响,占总耗时 70\% 以上;
\item 在安装阶段生成备份时,预制作 repo 在 SSD 场景有略微(10\% 左右)优化,在 HDD性能会退化;
\item 准备备份集合 rootfs 时间,在 SSD 上占比 10\% 左右,在 HDD 上 占比 20\%+。(潜在优化项);
\end{itemize}

\subsubsection{性能劣化消减方案}
\begin{itemize}[leftmargin=4em]
\item 安装器中添加是否生成初始化备份的选项,默认不勾选,减少默认状态下对安装时长的影响;
\item 数据回滚时,在待备份目录下创建临时目录,再将回滚目录与待备份目录进行比对,若相同则直接将待备份目录文件硬链接至临时目录下,从而减少回滚拷贝文件时间;
\end{itemize}

\subsection{安全性}
deepin 更新程序在数据安全,权限安全进行设计:

\subsubsection{数据安全}
\begin{itemize}[leftmargin=4em]
  \item 由于程序会访问修改系统文件,所以在更新程序中发生的任何操作都由管理员操作执行;
\end{itemize}

\subsubsection{权限安全}
\begin{itemize}[leftmargin=4em]
  \item 在程序提交和回滚系统文件时不会改变其权限,所以不会导致系统文件的权限问题;
\end{itemize}

\subsubsection{参数安全}
\begin{itemize}[leftmargin=4em]
  \item 在程序内部对恶意传参和参数错误传入都会进行比对,若程序内无此参数设置则会进行报错退出;
\end{itemize}

\subsection{可靠性}
\subsubsection{系统版本提交}
\begin{quote}
出现系统断电:
\begin{itemize}
  \item 利用 ostree 的提交功能,在未完全提交时中断并不会显示至提交列表中;
\end{itemize}
出现系统磁盘不足:
\begin{itemize}
  \item 更新程序的仓库存放至最大分区中,若其在根分区并空间不足时,会进行报错并退出;
  \item 在文件提交时会新建 rootfs 文件夹至与 /usr 同分区下,其内容大部分为硬链接,若空间不足时,会退出程序报错并将 rootfs 文件删除;
\end{itemize}
出现数据磁盘不足:
\begin{itemize}
  \item  当版本提交时,若数据磁盘空间不足,会进行报错退出并将新建 rootfs 文件进行删除;
\end{itemize}
\end{quote}

\subsubsection{系统版本回滚}
\begin{quote}
出现系统断电:
\begin{itemize}
  \item 每次仓库检出版本会将上次缓存进行删除,再重新 checkout;
  \item 在版本回滚文件替换时,会将引导文件最后替换防止由于系统断电,导致引导错误无法启动系统;
\end{itemize}
出现系统磁盘不足:
\begin{itemize}
  \item 更新程序的仓库存放至数据分区,会保证其足够大小,仓库回滚时首先在 /usr 同分区下新建 rootfs 文件夹用于存放回滚版本内容,若分区空间不足则会进行报错退出并将其删除;
\end{itemize}
\end{quote}
\subsection{兼容性}

  \subsubsection{V20的升级与回滚}
    \begin{itemize}
      \item 在 V20 升级 V23 时,若数据分区的大小不足无法存放仓库,则会自动选择最大分区进行存储;
      \item 在 V20 升级 V23 时,会将 init 程序进行升级,需其支持参数传入指定内核进行系统启动,防止系统回滚失效;
      \item V20 与 V23 交互详细流程在本文 2.3.2 详细说明,此功能为原子更新主要功能点;
    \end{itemize}
  \subsubsection{V20 兼容性问题}
  目前 V20 的运行兼容只考虑在 V20 与 V23 的升级回滚,暂不考虑 V20 本身的升级回滚,其不兼容问题如下;
  \begin{quote}
    规避出现运行系统环境不兼容问题:
    \begin{itemize}
      \item 更新程序为 go 开发,静态编译,运行时不依赖系统环境;
    \end{itemize}
    规避出现运行系统目录不兼容问题:
    \begin{itemize}
      \item 在 repo 构建时,只与目录相关联,系统的目录结构并未变更;
    \end{itemize}
    \end{quote}

\section{部署与实施}
\subsection{ISO 预装}
    在 V23 的 ISO 镜像中,需要把原子更新模块期望的系统目录进行集成,其在本文的 2.2.4 和 2.3.1 有详细说明;
\subsection{安装器初始化}
    在 安装器对 ISO 安装时,需要对已在ISO集成的期望目录进行循环解压,其在本文 2.3.3 有详细交互图说明;

\section{变更记录}
\subsection{V1.6}
  \subsubsection{详细信息}
  \begin{itemize}[leftmargin=4em]
    \item 修改章节 2.3.1 , 根据升级工具,补充v20升级v23逻辑;
    \item 修改章节 2.3.2 , 优化安装器与原子更新交互逻辑;
    \item 修改章节 2.3.3 , 增加设置默认配置(SetDefaultConfig)的逻辑流程图;
    \item 修改章节 2.3.4 , 修改仓库初始化逻辑,增加对配置文件的处理;
    \item 修改章节 2.3.6 , 增加提交时对配置文件备份处理;
    \item 修改章节 2.3.7 , 修改回滚逻辑,增加对配置文件的处理;
    \item 修改章节 2.4.3 , 增加dbus的 SetDefaultConfig 接口与 DefaultConfig, RepoUUID 属性;
    \item 修改章节 2.4.4 , 增加原子更新data.yaml 配置文件设计;
    \item 修改章节 2.4.6 , 将配置文件 config.json 路径修改为仓库路径,并增加存放对应版本配置的文件路径机制;
    \end{itemize}

\end{document}
